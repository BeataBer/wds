\documentclass[a4paper,12pt]{article}
\usepackage{polski}
\usepackage[T1]{fontenc}
\usepackage[utf8]{inputenc}
\usepackage[top=2cm, bottom=2cm, left=3cm, right=3cm]{geometry}
\usepackage{indentfirst}
\usepackage{enumerate}

\makeatletter
\newcommand{\linia}{\rule{\linewidth}{0.4mm}}
\renewcommand{\maketitle}{\begin{titlepage}  
    \vspace*{1cm}
    \begin{center}
  Wizualizacja danych sensorycznych - projekt
    \end{center}
      \vspace{3cm}
    \begin{center}
     \LARGE \textsc {\@title}
         \end{center}
     \vspace{1cm}
    
    \begin{center}
    \textit{ Autorzy:}\\
   \textit{\@author} 
     \end{center}
      \vspace{1cm}
    
    
    \vspace*{\stretch{6}}
    \begin{center}
    \@date
    \end{center}
  \end{titlepage}
}
\makeatother
\author{Beata Berajter 218629\\
Ada Weiss }%wpisać indeks
\title{Wizualizacja danych z czujników line followera}


\begin{document}
\newpage
\maketitle
\newpage
\tableofcontents

\newpage
\section{Opis projektu}
Założeniem projektu jest zebranie danych pobranych z czujników line followera. Czujniki, z których należy pobrać  informacje to czujniki optyczne oraz enkodery. Dane zostaną zebrane w celu ich wizualizacji. W projekcie zostanie umieszczona animacja, która pokaże rozmieszczenie czujników, oraz w przypadku transoptorów odbiciowych pokarze, który z nich aktualnie wykrył linię. Dane z czujników będą przetwarzane na wykresy oraz ilustracje "słupkowe". W celach  projektu zostało także założone, że narysowana zostanie droga/ścieżka, którą podąża robot.



\end{document}